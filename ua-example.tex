\documentclass[draft]{ua-thesis}

\director{Advisor's Name}
\author{The Graduate College}
\title{Manual for Theses and Dissertations formatted with
       \texttt{ua-thesis.cls} with the \texttt{draft} Option}
\date{1996}

\begin{document}

\maketitle

\chapter*{Acknowledgments}

The contents of this example dissertation has been entirely written by the
Graduate College at the University of Arizona.

\tableofcontents
\listoffigures
\listoftables

\begin{abstract}
This example dissertation contains the original text of the ``Manual for
Theses and Dissertations'', written by the Graduate College at the
University of Arizona.  It has been obtained via the internet at
\begin{quote}
http://grad.admin.arizona.edu/degreecert/ThesisManual/manual.htm
\end{quote}
on May 10, 1996.  The page was last updated November 9, 1995. No guarantee
is made that this information is current, and students should check with
the Graduate College before submitting a dissertation or thesis.
\end{abstract}

\chapter{Introduction}

Use this manual as a guide for setting up the physical format of your
thesis, dissertation or document. Your thesis will represent you, your
department, and The University of Arizona in the international scholarly
community. Your work is important and worthy of professional presentation.
This manual lists Graduate College requirements for the mechanical aspects
of meeting these high standards.

In this manual the word thesis, includes documents and dissertations. If
format requirements for the document or dissertation vary from those for the
thesis, specific requirements for each type of paper will be listed.

Two final copies of the thesis must be submitted; both must meet all
specifications of this manual. The two final copies should be submitted
unbound in a box to the Graduate College Degree Certification Office.

\chapter{University Microfilms Incorporated (UMI)}

Your thesis will be published by University Microfilms Incorporated, Ann
Arbor, Michigan. Upon certification by your major professor, your examining
committee, and the Graduate College, a copy of the thesis and a Special
Abstract are forwarded to UMI. The manuscript is cataloged and microfilmed,
the microfilm negative is inspected and put in vault storage. Paper copies
of your work will be produced on demand by UMI. Catalog information is sent
to the Library of Congress for production and distribution of catalog cards
for libraries. The original copy of the thesis is returned to The University
of Arizona Library. The Special Abstract is printed in Microfilm Abstracts
and distributed to leading libraries in the United States and abroad and to
a selected list ofjournals and abstracting services.

Publication by UMI does not preclude publication by other means later. You
are urged to submit your work for publication in a scholarly or professional
journal. Suitable acknowledgment must indicate that the publication is a
thesis, dissertation, or document, or portion thereof, which was submitted
in partial fulfillment of the requirements for a degree at the University of
Arizona.

You must complete a UMI publication agreement, available through the Degree
Certification Office.


\chapter{General Format Requirements}

\section{Margins}

Text, illustrations (figures) or tables must not appear outside the
specified margins. Specific margin requirements are listed in ORDER OF
SECTIONS under each category. Page numbers are the only item which may
appear outside the margin requirements.

\section{Corrections on Pages}

Do not use correction fluid or correction tape. These materials flake off in
handling and storage, exposing the original errors.

\section{Page numbers}

The title page is page 1 of the thesis. All pages which follow are numbered
in a single sequence with arabic numerals. Page numbers must be placed at
least 1 " below the top of the sheet, and 1" from the right edge. The
numbers must be at least 1/4" above the first line of text. You may omit the
printed page number on the title page; all other pages must have printed
page numbers. Do not use page headers. Do not use the phrase, Page xx; just
the numeral.

\section{Paper}

See Table \ref{t1}.

\begin{table}
\hrule
\begin{tabular*}{\textwidth}{@{}l@{\extracolsep{\fill}}l@{}}
  1. & Xerox Image Series Smooth Paper \\
  2. & Xerox Image Elite Paper 25\% Cotton \\
  3. & Xerox Series 10 \\
  4. & 100\% Cotton - 20 lb weight \\
  5. & Cranes Crest Bond - 100\% Cotton Acid Free \\
\end{tabular*}
\hrule
\caption{Required Paper} \label{t1}
\end{table}
  

\section{Photocopy Quality}

Photocopies must meet all requirements for margins, readability, and type of
paper. This includes all photocopied documents, tables, illustrations and
appendix pages.

\section{Printers}

Laser printing or other letter quality printing is required. Impact, or
daisy wheel printing is generally acceptable. 24-pin dot matrix near letter
quality and draft quality printing are not acceptable.

\section{Type Fonts}

Standard serif typefaces such as Courier and Times Roman reproduce and
microfilm well. Do not use modern Sans Serif types, which read well in the
original but do not reduce well for microfilming. Ornamental styles such as
Script and Old English may not be used. Limit the use of italic styles to
standard uses in bibliographic citations and foreign words. Boldface should
be restricted to very small segments of the text and to infrequent
occurrences.

\subsection{Type Size}

Use 12-point or 14-point for proportional fonts; 10 cpi or 12 cpi for
non-proportional fonts. A proportional font allows proportional spacing - a
feature that gives a printed page a more pleasing appearance by allowing for
different widths of characters. The letter w, for example, is wider than the
letter i. Normally, when these letters are printed, both are given the same
amount of space; the result can be gaps that are visually distracting. With
proportional printing, the letter w is given more space than the letter i,
creating a more aesthetic and professional-looking line of text.

\subsection{Typewritten Papers}

Papers prepared on good quality electric typewriters are acceptable. All
margin, paper quality, and typographic requirements apply. Type size should
be Pica (10 cpi) or Elite (12 cpi)


\chapter{Order of Sections}

Components of your thesis must be in the following order, formatted as
specified:

\begin{enumerate}
  \item Title Page
        \begin{itemize}
        \item Required
        \item Margins:
             \begin{itemize}
             \item Top 2.5"
             \item Bottom 1.5"
             \item Left 1.5"
             \item Right 1"
             \end{itemize}
        \item Spacing: Follow sample
        \end{itemize}
  \item Final Examining Committee Approval Form
        \begin{itemize}
        \item Required for dissertations and music documents, not for theses.
          The approval form for the thesis is included in the Statement by
          Author (see item 3 below).
        \item Note: Before the final oral defense the student obtains the
          Approval Pages from the Degree Certification Office. Original
          signatures are required on both final copies.
        \end{itemize}
  \item Statement by Author
        \begin{itemize}
        \item Required
        \item Margins:
             \begin{itemize}
             \item Top 2.5"
             \item Bottom 1"
             \item Left 1.5"
             \item Right 1"
             \end{itemize}
        \item Spacing: Single
        \item Note: Follow examples. Original signatures are required on both
          final copies.
        \end{itemize}
  \item Acknowledgements
        \begin{itemize}
        \item Optional
        \item Margins: Same as Body of Paper
        \item Spacing: Maybe single spaced
        \item Note: One page maximum
        \end{itemize}
  \item Dedication
        \begin{itemize}
        \item Optional
        \item Margins: Same as Body of Paper
        \item Spacing: Must be double spaced
        \item Note: One page maximum
        \end{itemize}
  \item Table of Contents
        \begin{itemize}
        \item Required
        \item Margins: Same as Body of Paper
        \item Note: See "Notes for Table of Contents" for notes on format.
        \end{itemize}
  \item List of Illustrations/List of Tables
        \begin{itemize}
        \item Required if document contains illustrations, figures or tables.
        \item Margins: Same as Body of Paper
        \item Note: Formatted like Table of Contents; see "Notes for List of
          Illustrations / List of Tables" for more information.
        \end{itemize}
  \item Abstract
        \begin{itemize}
        \item Required
        \item Margins:
             \begin{itemize}
             \item Top 1.5"
             \item Bottom 1"
             \item Left 1.5"
             \item Right 1"
             \end{itemize}
        \item Spacing: Double spaced
        \item Note: A Special Abstract for UMI is also required. The text
          remains the same for both versions, but formatting requirements
          differ. See "Abstract and Special Abstract Compared".
        \end{itemize}
  \item Body of Paper
        \begin{itemize}
        \item Required
        \item Margins:
             \begin{itemize}
             \item Top 1.5"
             \item Bottom 1"
             \item Left 1.5"
             \item Right 1"
             \end{itemize}
        \item Spacing: Double, except for long quotations, footnotes, table and
          illustration captions
        \item Note: Begin each major section on a new page. Margin requirements
          apply to every page of the thesis unless otherwise specified in
          this manual. See APPENDIX A, INCLUSION OF PUBLISHED PAPERS OR
          MANUSCRIPTS FOR PUBLICATION, if your department allows this
          option.
        \end{itemize}
 \item Appendices
        \begin{itemize}
        \item Optional
        \item Margins: Same as Body of Paper
        \item Spacing: Depends on nature of Appendix material
        \item Note: Each Appendix must begin on a new page.
        \end{itemize}
 \item References
        \begin{itemize}
        \item Required if citations are used
        \item Margins: Same as Body of Paper
        \item Spacing: Citations single spaced; double space between citations
        \item Note: Use your department's preferred citation style; consult with
          your advisor if more than one style is acceptable. Title this
          section REFERENCES or WORKS CITED. Do not use the word,
          Bibliography.
        \end{itemize}
\end{enumerate}

\section{Sample Title Page}

Margins:
\begin{itemize}
   \item Top 2.5"
   \item Bottom 1.5"
   \item Left 1.5"
   \item Right 1"
   \item The title page is centered within the left and right margins.
\end{itemize}
Title in capital letters:
\begin{verbatim}
               SEBASTIEN CASTELLIO, APOSTLE OF TOLERANCE

                       IN THE SIXTEENTH CENTURY
\end{verbatim}
Use your full name, spelled out:
\begin{verbatim}
                                  by

                          Jane Allison Smith
\end{verbatim}
This rule (solid line) is 2" long and is placed approximately 5" below the
top of the page. Copyright statement, if used, is placed directly below the
rule:
\begin{verbatim}
                       _____________________
                Copyright (c) Jane Allison Smith 1992
\end{verbatim}
Follow the capitalization and spacing of the lines in the bottom section:
\begin{verbatim}
             A Thesis Submitted to the Faculty of the

                 DEPARTMENT OF ROMANCE LANGUAGES

           In Partial Fulfillment of the Requirements
                       For the Degree of

                        MASTER OF ARTS
                   WITH A MAJOR IN SPANISH

                   In the Graduate College

                  THE UNIVERSITY OF ARIZONA

                           1 9 9 2
\end{verbatim}

\subsection{Notes}

This is a sample page for a THESIS. For a doctoral degree substitute
DISSERTATION for THESIS; for a musical arts degree, substitute DOCUMENT for
THESIS. Substitute DOCTOR OF PHILOSOPHY, DOCTOR OF MUSICAL ARTS, or other
title as appropriate.

The statement, WITH A MAJOR IN..., is included only if the name of the major
field of study is not exactly the same as the official name of the
department

Put spaces between the digits of the year: 1 (space) 9 (space) 9 (space) 2

In its entirety, see Figure \ref{f1}
\begin{figure}
\hrule
\begin{verbatim}

               SEBASTIEN CASTELLIO, APOSTLE OF TOLERANCE

                       IN THE SIXTEENTH CENTURY

                                  by

                          Jane Allison Smith

                       _____________________
                Copyright © Jane Allison Smith 1992

             A Thesis Submitted to the Faculty of the

                 DEPARTMENT OF ROMANCE LANGUAGES

           In Partial Fulfillment of the Requirements
                       For the Degree of

                        MASTER OF ARTS
                   WITH A MAJOR IN SPANISH

                   In the Graduate College

                  THE UNIVERSITY OF ARIZONA

                           1 9 9 2

\end{verbatim}
\hrule
\caption{Example Title Page}\label{f1}
\end{figure}


\subsection{Copyrighting the Thesis or Dissertation}

Copyrighting of a thesis is optional. Publication by University Microfilms
does not preclude publication by other methods later. If you want University
Microfilms Incorporated to file, on your behalf, an application for
registration of a claim of copyright on your manuscript, you must indicate
this on the agreement form you complete for UMI and submit the required fee
by certified check or money order. This service includes payment of the
registration fee, preparation of the application, and submission of copies
required by the Copyright Office.

The ownership of a copyright shall reside with the student unless otherwise
stated by University policy or by terms of the research grants, fellowships,
financial aid, etc. which were used to support the student's research.

Additional information on obtaining a copyright is available from the
Graduate College Degree Certification Office or the United States Copyright
Office, Library of Congress, Washington, D. C. 20559.


\section{Statement by Author}

Below are four samples of the Statement by Author. Select the applicable
sample: non-copyrighted thesis, copyrighted thesis, non-copyrighted
dissertation, or copyrighted dissertation, and copy it. Doctor of Musical
Arts students must substitute document for thesis or dissertation. For
theses, substitute the full name of your advisor for the name on the sample
page, Figures~\ref{f2}, \ref{f3}, \ref{f4}, \ref{f5}.

Margins:
\begin{itemize}
   \item Top 2.5"
   \item Bottom 1.5"
   \item Left 1.5"
   \item Right 1"
\end{itemize}


\begin{figure}
\hrule
\begin{verbatim}

                 STATEMENT BY AUTHOR

   This thesis has been submitted in partial fulfillment of
requirements for an advanced degree at The University of
Arizona and is deposited in the University Library to be
made available to borrowers under rules of the Library.

   Brief quotations from this thesis are allowable without
special permission, provided that accurate acknowledgment of
source is made.  Requests for permission for extended
quotation from or reproduction of this manuscript in whole
or in part may be granted by the head of the major
department or the Dean of the Graduate College when in his
or her judgment the proposed use of the material is in the
interests of scholarship.  In all other instances, however,
permission must be obtained from the author.

                     SIGNED: ________________________________

               APPROVAL BY THESIS DIRECTOR

  This thesis has been approved on the date shown below:

_________________________________  _________________________
         Jane M. Doe                        Date
   Professor of Chemistry

\end{verbatim}
\hrule
\caption{Non-copyrighted thesis} \label{f2}
\end{figure}

\begin{figure}
\hrule
\begin{verbatim}

                 STATEMENT BY AUTHOR

   This thesis has been submitted in partial fulfillment of
requirements for an advanced degree at The University of
Arizona and is deposited in the University Library to be
made available to borrowers under rules of the Library.

   Brief quotations from this thesis are allowable without
special permission, provided that accurate acknowledgment of
source is made.  Requests for permission for extended
quotation from or reproduction of this manuscript in whole
or in part may be granted by the copyright holder.

                     SIGNED: ________________________________

               APPROVAL BY THESIS DIRECTOR

  This thesis has been approved on the date shown below:

_________________________________  _________________________
         Jane M. Doe                        Date
   Professor of Chemistry

\end{verbatim}
\hrule
\caption{Copyrighted thesis} \label{f3}
\end{figure}

\begin{figure}
\hrule
\begin{verbatim}

                 STATEMENT BY AUTHOR

   This dissertation has been submitted in partial
fulfillment of requirements for an advanced degree at The
University of Arizona and is deposited in the University
Library to be made available to borrowers under rules of the
Library.

   Brief quotations from this dissertation are allowable
without special permission, provided that accurate
acknowledgment of source is made.  Requests for permission
for extended quotation from or reproduction of this
manuscript in whole or in part may be granted by the head of
the major department or the Dean of the Graduate College
when in his or her judgment the proposed use of the material
is in the interests of scholarship.  In all other instances,
however, permission must be obtained from the author.

                     SIGNED: ________________________________

\end{verbatim}
\hrule
\caption{Non-copyrighted dissertation} \label{f4}
\end{figure}


\begin{figure}
\hrule
\begin{verbatim}

                 STATEMENT BY AUTHOR

   This dissertation has been submitted in partial
fulfillment of requirements for an advanced degree at The
University of Arizona and is deposited in the University
Library to be made available to borrowers under rules of the
Library.

   Brief quotations from this dissertation are allowable
without special permission, provided that accurate
acknowledgment of source is made.  Requests for permission
for extended quotation from or reproduction of this
manuscript in whole or in part may be granted by the copyright
holder.

                     SIGNED: ________________________________

\end{verbatim}
\hrule
\caption{Copyrighted dissertation}\label{f5}
\end{figure}


\section{Notes for Table of Contents}

The Table of Contents is required. All levels of subheadings in your
manuscript must appear in the Table of Contents. The TABLE OF CONTENTS for
this Manual contains two levels; your table of contents might contain more.

Margins:
\begin{itemize}
   \item Top 1.5"
   \item Bottom 1"
   \item Left 1.5"
   \item Right 1"
\end{itemize}
Format requirements include:
\begin{itemize}
   \item The heading TABLE OF CONTENTS at the top of the first page of this
     section, and TABLE OF CONTENTS - Continued on each continuation page.
   \item Page numbers at upper right.
   \item Dot leaders .......................... from headings to page numbers.
   \item Indent each level of subheadings 4 spaces from the level above.
   \item Headings in the Table of Contents must exactly match the headings used
     in the body of the paper, and should be typographically the same (e.g.,
     type font and style, capitalization).
   \item Use all capital letters for major headings. (Subheadings may be upper
     and lower case.)
   \item Each Appendix must have its own letter designation and title.
     Appendices are major divisions. The title appears in caps on the left
     margin at the same level of importance as chapter headings.
\end{itemize}
Format options include:
\begin{itemize}
   \item Chapter numbering. You may number your chapters with either arabic or
     roman numerals.
   \item Subheading numbers. If chapters are numbered, you may also number
     subheadings.
\end{itemize}
For example, see Figures \ref{f6} and \ref{f7}.

\begin{figure}
\hrule
\begin{verbatim}

       I. LIST OF TABLES ...............................  6
      II. INTRODUCTION .................................  8
              Literature Review ........................  8
              Statistical Methods ...................... 11

\end{verbatim}
\hrule
\caption{Example 1 for Table of Contents} \label{f6}
\end{figure}

\begin{figure}
\hrule
\begin{verbatim}

       1. LIST OF TABLES ...............................  6
       2. INTRODUCTION .................................  8
             2.1 Literature Review .....................  8
             2.2 Statistical Methods ................... 11

\end{verbatim}
\hrule
\caption{Example 2 for Table of Contents} \label{f7}
\end{figure}

\section{Notes for List of Illustrations / List of Tables}

These lists, which resemble the Table of Contents, are required if your
thesis contains illustrations, figures, graphs or tables. Include a List of
Illustrations (or List of Figures) for figures, maps and drawings. Include a
List of Tables for graphs and tables. Illustrations or tables which appear
in the appendices only may or may not be included with the List of
Illustrations (or List of Figures) or the List of Tables.

Material in the List of Illustrations is numbered in sequence, Figure 1,
Figure 2, etc. You may construct this sequence as you wish, e.g., Figure
1.1, 1.2, 2.1, 2.2.... Use LIST OF ILLUSTRATIONS as the title for the first
page and LIST OF ILLUSTRATIONS - Continued for subsequent pages. You may use
LIST OF FIGURES instead of LIST OF ILLUSTRATIONS if your department prefers.

Material in the List of Tables should be given its own separate sequence of
numbers, Table 1, Table 2, etc. You may also construct this sequence as you
wish. Use LIST OF TABLES as the title for the first page and LIST OF TABLES
- Continued for subsequent pages.

For examples see Figures \ref{f8} and \ref{f9}.
\begin{figure}
\hrule
\begin{verbatim}

                          LIST OF ILLUSTRATIONS

FIGURE 1.1, Topographic map of valley ............................. 14
FIGURE 1.2, View of valley, northwest approach .................... 20
FIGURE 2.1, Cross section of burial mound ......................... 27
FIGURE 2.2, Intact jar found near mound ........................... 28

\end{verbatim}
\hrule
\caption{Sample List of Illustrations} \label{f8}
\end{figure}

\begin{figure}
\hrule
\begin{verbatim}

                               LIST OF TABLES

TABLE 1.1, Population estimates, 850-1100 a.d. .................... 16
TABLE 2.1, Number and type of artifacts compared with nearby sites  29

\end{verbatim}
\hrule
\caption{Sample List of Tables} \label{f9}
\end{figure}


\section{Abstract and Special Abstract Compared}

\subsection{Abstract}

The Abstract is included as a page of your thesis. It is a numbered page in
the thesis, appearing just before the main body of the text. The heading
ABSTRACT is centered at the beginning of the first page. A sample abstract
page follows (Figure~\ref{f10}). Compare it with the Special Abstract
sample.

Margins:
\begin{itemize}
   \item Top 1.5"
   \item Bottom 1"
   \item Left 1.5"
   \item Right 1"
\end{itemize}


\subsection{Special Abstract}

The Special Abstract contains the same text as the Abstract, but is
formatted for microfilming by UMI for Dissertation Abstracts International.
Treat the Special Abstract as a separate document. Include one copy of the
Special Abstract and two extra copies of your title page when submitting the
two final copies of your thesis.

Note the heading of the Special Abstract sample page which follows. Use the
thesis title as it appears on your title page. Use your full name, and add
the appropriate designation for your degree. Include your director's name as
shown (Figure~\ref{f11}).

Margins:
\begin{itemize}
   \item Top 1.5"
   \item Bottom 1"
   \item Left 1.5"
   \item Right 1"
\end{itemize}

\begin{figure}
\hrule
\begin{verbatim}

                          ABSTRACT

   This is the body of your abstract, limited to 150 words

for a thesis, and 350 words for a dissertation or document.

The word count limits apply to the regular Abstract in the

thesis and to this separate Special Abstract.  Use the same

text for both; just adjust the margins and heading.  The

abstract should summarize your work.  The UMI booklet listed

in the resources section of this Manual (here) provides some

writing tips.  The abstract for a dissertation or document

may be longer than one page; word count is more important

than page length in this section.

   This version of the Abstract, with simple heading, page

number, and 1.5" top margin, is included as part of your

thesis.

\end{verbatim}
\hrule
\caption{Sample Abstract} \label{f10}
\end{figure}

\begin{figure}
\hrule
\begin{verbatim}

                      COMPLETE TITLE OF

            YOUR THESIS, DISSERTATION OR DOCUMENT

                   Jane Allison Doe, M.S.

               The University of Arizona, 1992

Director: John J. Jones

   This is the body of your abstract, limited to 150 words

for a thesis, and 350 words for a dissertation or document.

The word count limits apply to the regular Abstract in the

thesis and to this separate Special Abstract.  Use the same

text for both; just adjust the margins and heading.  The

abstract should summarize your work.  The UMI booklet listed

in the resources section of this Manual (here) provides some

writing tips.  The abstract for a dissertation or document

may be longer than one page; word count is more important

than page length in this section.

   This version of the Abstract, with special heading, no

page number, and 1.5" top margin, is a separate document for

UMI.  Submit one copy of the special abstract, and two extra

copies of your title page, in the box with the final copies

of your thesis.

\end{verbatim}
\hrule
\caption{Sample Special Abstract} \label{f11}
\end{figure}

\appendix

\chapter{Inclusion of Published Papers or Manuscripts for Publication}

Under a policy adopted by The University of Arizona Graduate Council in
January, 1992, your department may allow published and publishable papers to
be included as part of your thesis. The reprints or manuscripts are treated
as appendices, and the body of your thesis must include a summary of your
contribution and a summary of the research. The Graduate College will accept
theses in this format from any unit with an implementation policy on file
with the Graduate College Degree Certification Office.

\section{Body of Paper}
 The ORDER OF SECTIONS applies. In addition, the Body of the
Paper must include two chapters as follows:
\begin{enumerate}
  \item An introduction describing the unique contribution of your work to the
     field of study. Use the following subsections as appropriate:
       \begin{enumerate}
       \item Explanation of the problem and its context
       \item A review of the literature
       \item Explanation of thesis format
          This subsection explains the relationship of the papers included
          and your contribution to each of the papers; where doctoral
          research efforts are part of a larger collaborative project, you
          must be able to identify one aspect of the project as your own and
          demonstrate an original contribution. Your role in the research
          and production of the published paper(s) should be clearly
          specified.
       \end{enumerate}
  \item A chapter titled PRESENT STUDY which summarizes the methods, results,
     and conclusions of the research. The chapter should begin with a
     statement such as:
\begin{quote}
          The methods, results, and conclusions of this study are
          presented in the papers appended to this thesis. The
          following is a summary of the most important findings in
          these papers.
\end{quote}
\end{enumerate}

\section{Appendices}
All mechanical requirements for Appendices listed in the ORDER
OF SECTIONS apply. Your appendices will consist of:
\begin{enumerate}
  \item A reprint of each paper as a separate appendix in the following
  order: 
       \begin{enumerate}
       \item a copy of the title page of the journal in which the article
          appeared
       \item the statement of permission for use of copyrighted material (see
          Appendix B: Permissions)
       \item the reprint(s), copied single-sided onto the required type of
          paper
       \end{enumerate}
  \item Supplementary materials such as data tables, graphs, and maps which
     might ordinarily appear as appendices to a thesis.
\end{enumerate}
These two types of appendices form a single sequence, assigned letters and
titled as described in this manual. All Appendix pages are part of the
single pagination sequence of the thesis. The page numbers will be typed in
as needed.


\chapter{Permissions}

Use of copyrighted material in your thesis, including illustrations, usually
requires written permission from the copyright holder. Start this
time-consuming process as early as possible. Play it safe and assume that
you must obtain permission if the material is copyrighted. Consult your
advisor or departmental graduate secretary about this process.

Exceptions, sometimes pertaining to small fractions of a musical score or
other document, are governed by the concept of "fair use." Factors weighed
in determining "fair use" include: the purpose of the use, whether
commercial or non-profit and educational; the nature of the copyrighted
work; the amount and substance of the material used in relation to the
entire work; and the effect of the use upon the potential market for or
value of the copyrighted work. The "fair use" concept is explained in detail
in the Chicago Manual of Style. According to the Association of American
University Presses, permission is required for quotations which are complete
units, for example, an entire poem, letter, book chapter, or an entire map,
chart, drawing or other illustration.

Permission to use copyrighted material should be in writing and retained by
the author. The release letters should indicate that permission extends to
microfilming and publication by University Microfilms Incorporated and that
the copyright owners are aware that UMI may sell, on demand, single copies
of the thesis, dissertation or document, including the copyrighted
materials, for scholarly purposes. UMI requires copies of permission letters
to be attached to the publication agreement, and assumes no liability for
copyright violations. If permission letters are not supplied, copyrighted
materials may not be filmed.

It is polite and good practice to obtain permission to use noncopyrighted
material, which may or may not be acknowledged in the text.

For additional information, telephone the Copyright Public Information
Office in Washington, DC, (202) 479-0700, weekdays between 8:30 a.m. and
5:00 p.m. EST or write to the Copyright Office, Library of Congress,
Washington D.C. 20559.

\chapter{Human/Animal Subjects Approval}

Research involving human subjects or live vertebrate animals requires
permission from the relevant University committee. Consult your research
director for details. If you are working on a project for which your
director has obtained the required permissions, be sure your name is listed
on the protocol approval and that you have the control number of the
approval in your records.

Research activities involving the use of human subjects require the review
and approval of the University Human Subjects Committee. A copy of the Human
Subjects approval letter along with The Human Subjects Research Statement
must be in the student's file in the Graduate College Degree Certification
Office. Questions regarding protocol can be answered by the Human Subjects
Committee. Their telephone number is (602) 626-6721.

Research involving any live vertebrate animals must be approved by the
Institutional Animal Care and Use Committee (IACUC) - The Animal Research
Protocol Review form must be completed by the student/instructor and
submitted to the protocol office for review and approval. Contact University
Animal Care for instructions, forms and protocol. Their telephone number is
(602) 621-3454

\chapter{Illustrations, Tables, Graphs}

Use illustrative material drawn or computer-generated in blick. Color will
reproduce in microfilm as shades of grey; use color only if essential to
convey a significant point in your work. Material may be laser-printed or
drawn in waterproof, permanent ink.
\begin{itemize}
   \item Use labels or symbols rather than color to identify lines on a graph
   \item Use cross-hatching rather than color to distinguish areas on a map
   \item Same margin requirements as Body of Paper
   \item Place the top of a horizontally-oriented page on the left; the page
     number should appear in the normal position (the upper right corner of
     the rotated sheet)
   \item Printed page numbers are required
\end{itemize}
If the caption is so long that it will not fit on the page with the
illustration or table, place it on its own numbered page immediately
preceding the page it describes.

\chapter{Oversized Materials}

Reduce oversized pages, such as maps and pictures, to 8.5 by 11 inches
without sacrificing legibility. If you must include oversized pages, two
options are available:
\begin{enumerate}
  \item Include a page 11 inches high, but wider than 8.5 inches, folded once
     or twice. The left edge of the foldout page should be even with the
     other pages of the thesis, and all folds should be made vertically. The
     folds on the right must be at least 1 inch from the right edge of other
     thesis pages to avoid damage to the foldout when pages are trimmed for
     binding. Place the page number in the upper right corner.

  \item If the oversized material cannot be reduced enough for the first
     option, roll it and submit in a sturdy mailing tube. The document must
     be labeled in the lower right corner with:
     \begin{itemize}
          \item Name
          \item Degree
          \item Department
          \item Title of thesis
          \item Date degree awarded
     \end{itemize}
     Use a separate tube for each copy and include the label information on
     the outside of each tube. When you make arrangements with any business
     to have copies of your thesis bound for departmental or private use,
     discuss procedures concerning folding of the oversize material and any
     additional fees which may be incurred for this service. In the LIST OF
     ILLUSTRATIONS use the phrase, in pocket, instead of a page number.
\end{enumerate}

\chapter{Photographs}

Photographs should be high contrast, low gloss blick and white pictures.
They may be printed on 8.5 x 11 inch photographic paper in order to avoid
mounting.

If mounting of small photographs on a standard page is necessary, use
double-sided transparent tape or photographic dry mounting tissue and mount
the photographs on \#67 pliable bristol board. High quality screen prints
or photocopies are acceptable.



\begin{thebibliography}{9}

\bibitem{A}
CCIT, Center for Computing \& Information Technology, offers workshops and
lectures on word processing, graphs and tables, and thesis/dissertation
formatting.

\bibitem{B}
Your advisor and your committee can help; providing assistance on thesis
formatting, a part of good mentoring.

\bibitem{C}
Doctoral Candidates: A Handbook For Completing the Steps To Your Degree.
Available at no cost from the Graduate College Degree Certification Office.

\bibitem{D}
Master's and Specialists Candidates: A Handbook For Completing the Steps To
Your Degree. Available at no cost from the Graduate College Degree
Certification Office.

\bibitem{E}
Publishing Your Dissertation: How to Prepare Your Manuscript for
Publication. UMI Dissertation Services. Available at no cost from the
Graduate College Degree Certification Office.

\bibitem{F}
Publishing Your Masters Thesis: How to Prepare Your Manuscript for
Publication. UMI Dissertation Services. Available at no cost from the
Graduate College Degree Certification Office.

\bibitem{G}
Publication style manuals for disciplines are available at the Main Library.

\end{thebibliography}
\end{document}
